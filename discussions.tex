\documentclass[10pt]{article}

%---------------------------------------------------------------------
\usepackage[a4paper, headsep=-4in,bindingoffset=0in,%
left=2.5cm,right=2.5cm,top=2.5cm,bottom=2.5cm,%
footskip=.25in]{geometry}
\newcommand{\textBF}[1]{%
    \pdfliteral direct {2 Tr 0.3 w} %the second factor is the boldness
     #1%
    \pdfliteral direct {0 Tr 0 w}%
}
\usepackage{multirow}
\usepackage{soul}

\newcommand{\beginsupplement}{%
\setcounter{table}{0}
\renewcommand{\thetable}{S\arabic{table}}%
\setcounter{figure}{0}
\renewcommand{\thefigure}{S\arabic{figure}}%
}

\def\Plus{\texttt{+}}
%\usepackage[english]{babel}   
\usepackage[utf8]{inputenc}  
\usepackage[font=scriptsize]{caption}
\usepackage{tabularx}
%\DeclareCaptionFont{6pt}{\fontsize{6pt}{6pt}\selectfont}
\captionsetup[figure]{font={stretch=1}}  
%\usepackage{sectsty}
\usepackage{subcaption}
\usepackage{wrapfig}
\usepackage{layout}
\usepackage{graphicx}
\usepackage{verbatim}
\usepackage{listings}
\usepackage{mathptmx}

\usepackage{booktabs}
\usepackage{etoolbox}


\usepackage{lmodern}
\usepackage[T1]{fontenc}
\usepackage[backend=biber,style=apa,sorting=none]{biblatex}
\addbibresource{paperpile.bib}
%\pagenumbering{gobble}
\pagenumbering{arabic}

\graphicspath{{figs/}}
\setlength{\topmargin}{-10pt}
%\renewcommand{\baselinestretch}{1.5}

\usepackage{indentfirst}
\setlength{\parindent}{1cm}
\usepackage[table]{xcolor}

\setlength{\headsep}{1pt}
%---------------------------------------------------------------------

\begin{document} 
\subsection*{Discussion}

In this paper we compare lesion biomarkers of chronic motor impairment head-to-head in the largest sample of stroke subjects to date. 

\subsubsection*{Why do ChaCo scores perform better than lesion load metrics?}

Regions that are not related to hemiparesis may have predictive benefits. ChaCo scores represent the structural disconnection of all brain regions and thus have full coverage to capture statistical associations between lesion location and motor impairment across the entire brain. Thus, ChaCo scores may simply have sufficiently fine resolution 




Right hemisphere has more areas that contribute to the prediction. Possibly complex relationship between lesion and deficit with right-handers when the lesion affects the non-dominant hand.




\subsubsection*{}
In \cite{Pineiro2000-dv} original CST mask paper; looked at cross sectional area of CST occupied by stroke and related it to motor score at 1 month, $r^2 = 0.82$.

In \cite{Jang2008-ns} pontine infarct subjects (25) with CST damage (from DTI) in acute stage have worse chronic motor scores. 

In \cite{Zhu2010-qh}, CST-ll observed at chronic stage is predictor of chronic stroke motor impairment in moderate to severe stroke, not lesion size per se. Regression analyses with $R^2 = 0.72$ for weighted lesion load, 50 patients. Weighted lesion load is accounting for the narrowing of the CST as it descends; slicewise adjustment. 

Similarly in \cite{Lindenberg2010-pa}, descending white matter tract damage (anterior and posterior descending primary motor pathways, not just PT) predicts chronic outcome.

In \cite{Lam2018-xh}, they assess the ability of the CST lesion load to predict chronic CMSA-motor and ARAT scores. CST ll accounted for 23$\%$ and 24$\%$ of the variance in motor scores, in 27 subjects with upper limb impairments.

In \cite{Feng2015-du} wCST-LL is a good predictor of 3 month FM outcome in subjects with severe initial injury ($R^2 = 0.47$).

In \cite{Lin2019-hy}, CST injury taken from acute imaging explained ~20 percent of the variance in the magnitude of REALIZED upper extremity recovery (i.e. recovery adjusted for the different extents of recovery based on initial impairment) at the chronic stage.

In \cite{Ito2022-em}, CST-ll is calculated separately for descending tracts from M1, S1, SMA, preSMA, PMv, and PMd; roughly 50 percent of CST descending fibers are from premotor areas. Use SMATT toolbox. PMv and M1 classify FM severity groups better than M1 alone; PMv has less collinearity with other tracts.

In \cite{Hayward2022-hv}, WM microstructure of the CC (FA) relates to motor impairment in severe and mild/moderate. 

\subsection*{Predicting chronic motor scores from structural disconnection: alternative measures}
In \cite{Dulyan2021-jf}, they predict 3 month and 1 year motor outcomes for L and R deficits separately by using disconnectome components. Show a median/mode R of 0.56 and 0.5 for 3 months and 1 year prediction respectively (with permutations). If we're just going to square the R values and call that explained variance, they get 31 percent explained variance for 1 year post stroke left motor deficits, and 4 percent explained variance for 1 year post stroke right motor deficits. The methods of this preprint are questionable, but they do evaluate their model on held-out data.

In \cite{Peters2018-tf}, argue that Wallerian degeneration (WD) which is detected as early as 2 weeks post stroke, produces changes to cortex beyond the lesion site and may be predictive of chronic motor impairments. CST fibers originate from the premotor and primary motor areas. Subcortical areas like the thalamus and red nucleus are also implicated in movement. Cortico-subcortical connectivity may be related to motor behaviour. Looked at lesion overlap with cortical and subcortical areas. Also did DTI tracking to get connectivity strength between M1, PMC, SMA. SC between ipsilesional M1/SMA and thalamus related to motor scores.  

In \cite{Salvalaggio2020-pe} voxelwise SDC, then PCA and ridge regression and LOOCV. Predictions of motor scores 2 weeks post-stroke from SDC give an $R^2 = 0.37$.

In Rondina et al., 2017...
In \cite{Rondina2016-ds} they show that theory-driven feature selection, including areas from the premotor/frontal areas, better predicts motor outcome than just looking at CST-related features. Voxels (lesion voxels) associated with chronic motor performance (via mass univariate voxel GLMs) not found in CST but in pre and post central gyrus.

In \cite{Sperber2021-lw} and \cite{Bzdok2020-py} they discuss the difference between inference and prediction.  \cite{Sperber2021-lw} demonstrate that the areas that may be useful for prediction of motor deficits after stroke may not be clinically meaningful. Related to \cite{Mah2014-cb}, showing how patterns of damage from lesion-symptom mapping are biased/shifted due to vasculature.
\end{document}